\documentclass{article}
\usepackage{graphicx}
\usepackage{hyperref}
\usepackage{import}
\usepackage{subfig}
\usepackage{longtable}
\usepackage
[
        a4paper,
        vmargin=2cm,
        top=2.3cm,
        bottom=3.5cm,
]
{geometry}


\makeatletter 
\def\@maketitle{
\raggedright

\begin{center}
{
\includegraphics[width = 30mm]{img/mc-logo-nobg.png}\\[3ex]
\huge \bfseries \sffamily \@title }\\[4ex] 
{\Large  \@author}\\[4ex] 
\@date\\[8ex]
\end{center}}
\makeatother

\title{\textbf{Plagiarism Report for %(title)s}}
\author{%(course)s}
\date{\today}

\begin{document}
\maketitle
\tableofcontents
\newpage

\section{Statistics}
In the analysis of the %(numofsub)s submitted files, a total amount of %(numdep)s relevant dependencies between
files were generated. %(numofplag)s of which were flagged as potentially plagiarized. Plagiarism was only
assessed on code for this submission as well as the baseline adjustments were only done on code.
The parameters for detecting plagiarism were set as follows:
\begin{table}[h]
\centering
\begin{tabular}{ll}
                               & \textit{value} \\ \hline
\textit{Plagiarism Threshold} & %(plagthresh)s   \\
\textit{Threshold Barren$\ast$}    & %(barrthresh)s
\end{tabular}
\end{table}\\
{\small $\ast$ If applicable: Threshold for discarding submissions if they are too similar to.}
%(distrsect)s
\\
To further analyze the potentially fraudulent dependencies a network graph is plotted, showing the similarity of code between individual submissions \hyperref[fig:ntwrk]{(Fig. \ref{fig:ntwrk}}). The names of the nodes correspond to the student number, with the value on edges being the similar to another submission.
The edge's thickness and color are adjusted based on the magnitude of their values. All cases are listed in \hyperref[sec:cl]{Case Listings}\\
\begin{figure}[h]
    \centering
    \includegraphics[width=\textwidth]{img/network.png}
    \caption{Network Graph for the Potential Cases of Plagiarism}
    \label{fig:ntwrk}
\end{figure}

\section{Case Listings}\label{sec:cl}
The columns\textit{ markdown} and \textit{code} depict the respective similarity values for each metric whereas \textit{combined} is $\sqrt{markdown^2+code^2}$.
%(cldiscifcodeonly)s \\
%(casetable)s

\newpage
\section{Cases}

\end{document}
