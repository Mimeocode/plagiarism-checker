\documentclass{article}
\usepackage{graphicx}
\usepackage{hyperref}
\usepackage{pythonhighlight}
\usepackage{import}
\usepackage{subfig}
\usepackage{longtable}
\usepackage
[
        a4paper,
        vmargin=2cm,
        top=2.3cm,
        bottom=3.5cm,
]
{geometry}


\makeatletter 
\def\@maketitle{
\raggedright

\begin{center}
{
\includegraphics[width = 30mm]{img/mc-logo-nobg.png}\\[3ex]
\huge \bfseries \sffamily \@title }\\[4ex] 
{\Large  \@author}\\[4ex] 
\@date\\[8ex]
\end{center}}
\makeatother



\title{\textbf{Plagiarism Report for %(title)s}}
\author{%(course)s}
\date{\today}

\begin{document}

\maketitle
\tableofcontents
\newpage

\section{Statistics}
In the examination of the %(numofsub)s submitted files, a comprehensive evaluation yielded a total of %(numdep)s pertinent interdependencies among files.
Among these, %(numofplag)s were identified and flagged as potential instances of plagiarism.
It is noteworthy that the evaluation exclusively focused on code segments within the submissions, and any baseline adjustments were confined solely to code-related elements.
The criteria and parameters used for the detection of plagiarism were configured as follows:
\begin{table}[h]
\centering
\begin{tabular}{ll}
                               & \textit{value} \\ \hline
\textit{Plagiarism Threshold} & %(plagthresh)s   \\
\textit{Threshold Barren$\ast$}    & %(barrthresh)s
\end{tabular}
\end{table}\\
{\small $\ast$ If applicable: The threshold for disqualifying submissions due to excessive similarity to}
%(distrsect)s
\\
To conduct a more thorough examination of potentially fraudulent dependencies, a network graph has been constructed to illustrate the code similarity among individual submissions (\hyperref[fig:ntwrk]{Fig. \ref{fig:ntwrk}}).
Within this graphical representation, nodes are labeled with corresponding student numbers, and the edges are weighted to signify the degree of similarity with other submissions.
The thickness and coloration of these edges are dynamically adjusted to convey the strength of their similarity scores.
It is crucial to note that, for this network graph, only those dependencies exceeding a similarity score threshold of %(cutoff)s are visually depicted. Comprehensive details for all cases can be found in the \hyperref[sec:cl]{Case Listings} section.
\begin{figure}[h]
    \centering
    \includegraphics[width=\textwidth]{img/network.png}
    \caption{Network Graph for the Potential Cases of Plagiarism}
    \label{fig:ntwrk}
\end{figure}

\section{Case Listings}\label{sec:cl}
The columns denoted as \textit{markdown} and \textit{code} present the individual similarity values for each metric, while the \textit{combined} column offers a comprehensive assessment through the calculation of the Euclidean distance, represented as $\sqrt{markdown^2 + code^2}$.
%(cldiscifcodeonly)s \\
%(casetable)s

\newpage
\section{Cases}
%(cases)s

\end{document}
